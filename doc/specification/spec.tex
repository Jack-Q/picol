%!TEX program = xelatex
%!TEX option = -shell-escape -8bit
\documentclass[a4paper, 11.5pt]{article}
\usepackage[margin={2cm,1.9cm},includefoot]{geometry}
\usepackage{amsmath} % Package for math formula, symbols, etc
\usepackage{amsfonts} % Package for extra math font package
\usepackage{tikz} % Package for vector graphic drawing
\usepackage{csquotes} % Package for qouted fragment
  \usepackage{fontspec}
  \usepackage{hyperref} % use this to support hyper-liniks
  \hypersetup{
    colorlinks=true,
    linkcolor=gray,
    filecolor=magenta,
    urlcolor=gray,
  }
  \urlstyle{same}
  \usepackage[skins,breakable]{tcolorbox}
  \usepackage{minted}
  \usemintedstyle{bw} % use bw style since the asm code can't be well-highlighted
  \usepackage{xcolor}
  \renewcommand\theFancyVerbLine{\textcolor{gray}{\normalsize\arabic{FancyVerbLine}}}
\usepackage{multicol}
\usepackage{lineno}
\setlength{\columnsep}{0.5cm}
\usepackage{enumitem}
\usepackage{syntax}
\setlength{\grammarparsep}{20pt plus 1pt minus 1pt} % increase separation between rules
\setlength{\grammarindent}{8em} % increase separation between LHS/RHS 
\usepackage{rail/rail}
\setlength{\parskip}{15pt}
\setlength{\parindent}{1em}
\setmainfont{Times New Roman}
\setmonofont{Monaco}


\title{Picol Language Specification}
\author{Jack Q}
\begin{document}
\maketitle

  \begin{multicols}{2}

  \section{Introduction}
  {\bfseries Picol} (\emph{Pico-Language}) is a small language for demonstrative purpose. 
  The name of Picol follows its simple and limited feature.

  \section{Syntax}

  \subsection{Overview}

  Keywords: the following words are preserved for use by language.
  


  $$\begin{array}{rrl}
    Statement & \to & \{ StatementSequence \}  \\
              &   | & Function                 \\
              &   | & Declaration;             \\
              &   | & Expression;              \\
              &   | & return Expression;       \\
              &   | & break;                 \\
              &   | & continue;                 \\
              &   | &                  \\
              &   | & Function                 \\
              &   | & Function                 \\
              &   | & Function                 \\
              &   | & Function                 \\
              &   | & Function                 \\
              &   | & Function                 \\
              &   | & Function                 \\
              &   | & Function                 \\
    
    Function  & \to & TypeExpr \cdot Id \cdot ( \cdot ArgList \cdot ) \cdot \{ \cdot StatSeq \cdot \} \\
  \end{array}$$

\begin{grammar}
  <statement> ::= `(' <statement> `)'
             \alt <function>
             \alt `return' <expression>
\end{grammar}
\begin{grammar}
 <statement> ::= `(' <statement> `)'
             \alt <function>
             \alt `return' <expression>
\end{grammar}

% This sections contains the language expressed as a specific BNF syntax
% The ``rail'' environment will abstract the content 
\begin{rail}
  statement : '{' statement-sequence '}'
            | function-declaration
            | 
            ;

  statement-sequence : statement statement-sequence
                     | statement;
  
  function-declaration : ;
  expression : term | expression '+' term;
  term       : factor | term '*' factor;
  factor     : constant | variable | '(' expression ')';
  variable   : 'x' | 'y' | 'z' ;
  constant   : digit | digit constant;
  digit      : '0' | '1' | '2' | '3' | '4' | '5' | '6' | '7' | '8' | '9';
\end{rail}

  \end{multicols}

\end{document}