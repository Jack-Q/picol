%!TEX program = xelatex
%!TEX option = -shell-escape -8bit
\documentclass[a4paper, 11.5pt]{article}
\usepackage[margin={2cm,1.9cm},includefoot]{geometry}
\usepackage{amsmath} % Package for math formula, symbols, etc
\usepackage{amsfonts} % Package for extra math font package
\usepackage{tikz} % Package for vector graphic drawing
\usepackage{csquotes} % Package for qouted fragment
  \usepackage{fontspec}
  \usepackage{hyperref} % use this to support hyper-liniks
  \hypersetup{
    colorlinks=true,
    linkcolor=gray,
    filecolor=magenta,
    urlcolor=gray,
  }
  \urlstyle{same}
  \usepackage[skins,breakable]{tcolorbox}
  \usepackage{minted}
  \usemintedstyle{bw} % use bw style since the asm code can't be well-highlighted
  \usepackage{xcolor}
  \renewcommand\theFancyVerbLine{\textcolor{gray}{\normalsize\arabic{FancyVerbLine}}}
\usepackage{multicol}
\usepackage{lineno}
\setlength{\columnsep}{0.5cm}
\usepackage{enumitem}
\usepackage{syntax}
\setlength{\grammarparsep}{20pt plus 1pt minus 1pt} % increase separation between rules
\setlength{\grammarindent}{8em} % increase separation between LHS/RHS 
\usepackage{rail/rail}
\setlength{\parskip}{15pt}
\setlength{\parindent}{1em}
\setmainfont{Times New Roman}
\setmonofont{Monaco}


\title{Picol Language Specification}
\author{Jack Q}

% macro for the name of the programming language
\newcommand{\picol}[0]{{\bfseries Picol}}
% inline code format
\newcommand{\ci}[1]{\mintinline[escapeinside=||]{text}{#1}}

\begin{document}
\maketitle

\begin{multicols}{2}

  \section{Introduction}
    \picol{} (\emph{Pico-Language}) is a small language for demonstrative purpose. 
    The name of \picol{} follows its simple and limited features.

    \picol{} supports basic arithmetic operations, relational operations and logical operations 
    on four primitive data types: integer (\ci{int}), float point number (\ci{float}), 
    character (\ci{char}) and boolean value (\ci{bool}). Arrays in \picol are collection of primitive values, 
    which in \picol{} can be either single-dimensional or multi-dimensional and the size of each 
    dimension can be specified at execution time. \picol also supports functions for code reuse and 
    organization, which can be nested within each other and invocated in a recursive manner.
    
    As an simple language, \picol{} has no support for multi-file organization of source code, 
    all code are stores in an single plain text file; \picol provides no facilities 
    for importing functionalities from other languages, yet a limited number of build-in functions 
    provides fundamental input/output capabilities and mathematical functions for this language.

    This document provides an rough description of features and details of languages,
    including its data types, syntax, execution model, build-in functions, 
    virtual runtime environment and implementation notes.

  \section{Data Types}
    \picol{} supports four primitive data types, which are integer (\ci{int}), float point number (\ci{float}), 
    character (\ci{char}) and boolean value (\ci{bool}). It also supports two derived data types,
    which are array type of primitive data type and reference type to array of primitive data type.
    Besides, a special \ci{void} type can be treated as a primitive data type in certain conditions.

    \subsection{Primitive Data Types}
      \subsubsection{integer (\ci{int})}
      \subsubsection{float point number (\ci{float})}
      \subsubsection{character (\ci{char})}
      \subsubsection{boolean (\ci{bool})}

    \subsection{Derived Data Types}
      \subsubsection{array}
      \subsubsection{reference to array}

    \subsection{Other Data Types}
      \subsubsection{\ci{void}}
  \section{Syntax}

    \subsection{Overview}

    Keywords: the following words are preserved for use by language.
   
    \begin{center}
      \begin{tcolorbox}[breakable, blanker, width=0.8\linewidth]
        \begin{minted}[autogobble, obeytabs, resetmargins, fontsize=\footnotesize]{javascript}
        return while if else switch case do 
        continue break default
        \end{minted}
      \end{tcolorbox}
    \end{center}

    \begin{grammar}
      <statement> ::= `(' <statement> `)'
                \alt <function>
                \alt `return' <expression>
    \end{grammar}
    \begin{grammar}
    <statement> ::= `(' <statement> `)'
                \alt <function>
                \alt `return' <expression>
    \end{grammar}

    \end{multicols}

    \begin{center}
        % This sections contains the language expressed as a specific BNF syntax
        % The ``rail'' environment will abstract the content 
        % This process can be finished via the shell script helper in this folder
        \begin{rail}
          Statement : '\{' StatementSequence '\}'
                    | FunctionDeclaration
                    | 'return' Expression ';'
                    | Expression ';'
                    | Declaration ';'
                    | 'if' '(' Expression ')' Statement
                    | 'if' '(' Expression ')' Statement 'else' Statement
                    | 'break' ';'
                    | 'continue' ';'
                    | 'switch' '(' Expression ')' '\{' OptionList '\}'
                    | 'do' Statement 'while' '(' Expression ')' ';'
                    | 'while' '(' Expression ')' Statement
                    ;      
        \end{rail}
    \end{center}
      
    \begin{multicols}{2}

    \begin{rail}
      StatementSequence : Statement StatementSequence | ();

      FunctionDeclaration : Type Identifier '(' ParameterList ')' \\
                          '\{' StatementSequence '\}';
      
      Declaration : PrimitiveType DeclareList | ArrayType Identifier;

      DeclareList : DeclareItem | DeclareItem ',' DeclareList;

      DeclareItem : Identifier | Identifier ':=' Expression;

      ParameterList : () | Type Identifier | Type Identifier ',' ParameterList ;
      Type: PrimitiveType | ArrayType;
      ArrayType : Type '[' ArrayDimension ']';
      ArrayDimension : IntegerLiteral | IntegerLiteral ',' ArrayDimension;
      PrimitiveType : 'int' | 'float' | 'char' | 'bool';

      OptionList : 'case' Expression ':' \\
                    Statement OptionList
                | 'default' ':' Statement
                | ()
                ;
    \end{rail}
    
  
    \begin{rail}      
      Expression : LeftValue AssignOperator Expression
                  | Expression BinaryOperator Expression
                  | LeftValue SelfModification
                  | SelfModification LeftValue
                  | LeftUnaryOperator Expression
                  | '(' Expression ')'
                  | Identifier '(' ArgumentList ')'
                  | LeftValue
                  | ConstantLiteral
                  ;
      LeftValue : Identifier
                | LeftValue '[' ArgumentList ']'
                ;
      AssignOperator : ':=' | '+=' | '-=' | '*=' | '/=';
      BinaryOperator : '+' | '-' | '*' | '/' 
                      | '>' | '>=' | '<' | '<=' | '=' | '!=' 
                      | '\&\&' | '||';
      SelfModification : '++' | '--';
      LeftUnaryOperator : '!' ;
      ArgumentList : () | Expression ArgumentListMore;
      ArgumentListMore : () | ',' Expression ArgumentListMore;

    \end{rail}

    \begin{rail}
      Identifier : IdentifierStart IdentifierMid;
      ConstantLiteral : IntegerLiteral | CharLiteral | FloatLiteral | BooleanLiteral;
      IdentifierMid: () | Digit IdentifierMid | IdentifierStart IdentifierMid;
      IdentifierStart: Letter | '\_';
      IntegerLiteral: '0' | DigitNonZero IntegerLiteralMid;
      IntegerLiteralMid : () | Digit IntegerLiteralMid;
      FloatLiteral: IntegerLiteral '.' IntegerLiteralMid;
      BooleanLiteral: 'true' | 'false';
      CharLiteral: "'" '$Ascii Chars$' "'";
      Letter : 'a $\ldots$ z' | 'A $\ldots$ Z';
      Digit : '0' | DigitNonZero ;
      DigitNonZero : '1 $\ldots$ 9';
    \end{rail}

  \section{Execution Model}

  \section{Build-in Functions}

  \section{Virtual Runtime Environment}

  \section{Implementation Notes}

  \section{Summary}

\end{multicols}

\end{document}